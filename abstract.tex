% +--------------------------------------------------------------------+
% | Copyright Page
% +--------------------------------------------------------------------+

\newpage

\thispagestyle{empty}

\begin{center}

{\bf \Huge Abstract}

  \end{center}
\vspace{1cm}

This work consist on the implementation of EPFIOT (Edge Provisioning For Internet of Things). An appliance developed to look at Infrastructure as a service based on the edge. It is specifically prepared for interacting with IoT devices using special hardware accelerators designed for AI.

Cloud computing is the only choice nowadays, certain doubts arise about if it would be mandatory to keep all device data sended directly to cloud assuming the latency costs. Epfiot relies in using low specs hardware trying to bring to your local area network a similar cloud model, simplifying the way of providing virtual machines through the use of real hardware accelerators for your devices. This will allow the device to perform inference over the collected data quickly.

The application has a graphql interface building an entire IoT ecosystem, using infrastructure on the edge thanks to Linux virtualization (kvm) and emerging technologies, like for example LwM2M to provide device bootstrapping.

\vspace{1cm}

% +--------------------------------------------------------------------+
% | On the line below, replace Fecha
% |
% +--------------------------------------------------------------------+

\begin{center}

{\bf \Large Keywords}

   \end{center}

   \vspace{0.5cm}
   
Internet of things, Virtualization, Infrastructure as a Service, IaaS, Edge computing, cloud computing, LWM2M, LAN, local area network, appliance, router, Virtual Machine
   


