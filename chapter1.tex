% +--------------------------------------------------------------------+
% | Sample Chapter
% |
% | This file provides examples of how to
% | - insert a figure with a caption
% | - construct a table with a caption
% | - create subsections within the chapter
% | - insert a reference to a Figure or Table
% | - make a citation
% +--------------------------------------------------------------------+

\cleardoublepage

% +--------------------------------------------------------------------+
% | Replace "Chapter Title" below with the title of your chapter.  LaTeX
% | will automatically number the chapters.
% +--------------------------------------------------------------------+

\chapter{Introducción}
%\label{ch:Introducción}
\label{makereference}


% +--------------------------------------------------------------------+
% | Replace \section headings below with the title of your
% | subsections.  LaTeX will automatically number the subsections 1.1,
% | 1.2, 1.3, etc.
% +--------------------------------------------------------------------+

\section{Motivation}
\label{makereference1.1}

Cloud is a major word speaking about data compute in the last decade, we use cloud computing for processing every chunk of information from any source. 
If we move to Internet of Things field it is impossible not to put cloud computing in the same topic, cloud services are designed to provide easy, scalable access to applications, resources and services, and are fully managed by a cloud services provider.~\cite{cloud_def}  

Internet of things devices provide large amount of data, and we need cloud to process such kind of data, the Cloud of Things (CoT) term comes into play having devices connected directly to the cloud for perform complex operations with the generated information.
However cloud computing now faces several challenges to meet the more stringent performance requirements of many applications services, specially in terms of latency and bandwith.~\cite{IEE:Morabito:2017}

To solve those problems, Edge computing is a new paradigm that aims to bring data storage and computation closer to a selected location in order to improve response times and save data bandwitch, this consist increasing the resources available on the edge adopting platform that provide intermediate layers.

Thinking about these layers located at the edge make impossible the idea of having the same datacenters that are used for Cloud computing, this implies limited computational capabilities so one of the main problems to solve is to get similar cloud platforms into an Edge environment with certain compute power for processing data. 

To achieve those goals we can leverage on the next concepts:

\begin{itemize}
  \item One entry in the list
  \item Another entry in the list
\end{itemize}



\section{Objetivos}
\label{makereference1.2}

In this section, we refer back to text mentioned in
Section~\ref{makereference1.1} on page~\pageref{makereference1.1}.

\section{Estructura de la Memoria}
\label{makereference1.3}

Here's an example of a citation to a single
work.~\citet{CT:Weiner:1999} It's also possible to make multiple
citations.~\citet{CT:Phillips:1985, ARP:Loy:1974}
