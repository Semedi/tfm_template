% +--------------------------------------------------------------------+
% | Copyright Page
% +--------------------------------------------------------------------+

\newpage

\thispagestyle{empty}

\begin{center}

{\bf \Huge Resumen en castellano}

  \end{center}
\vspace{1cm}

En este presente trabajo se realiza la implementación de EPFIOT (Edge Provisioning For Internet of Things). Una appliance desarrollada para contemplar casos de Infraestructura como servicio en el edge computing y específicamente preparada para interactuar con dispositivos IoT usando aceleradores hardware para IA.

En un mundo gobernado por el cloud computing, surgen dudas sobre si de verdad es necesario mandar todos los datos que recogen los dispositivos a internet afrontando la latencia que esto supone, Epfiot es un proyecto que intenta usando pocos recursos ofrecerte esa nube de infraestructura para procesar los datos de tus dispositivos en tu propia red local, facilitándote maquinas en el borde con aceleradores IA para que tus dispositivos realicen inferencia sobre los datos rápidamente.

La aplicación ofrece una interfaz graphql para que puedas montar un ecosistema IoT de infraestructura en el borde usando la virtualización de Linux (kvm) y otras tecnologías emergentes como puede ser LwM2M para facilitar la configuración de los dispositivos.

\vspace{1cm}

% +--------------------------------------------------------------------+
% | On the line below, repla	ce Fecha
% |
% +--------------------------------------------------------------------+

\begin{center}

{\bf \Large Palabras clave}

   \end{center}

   \vspace{0.5cm}
   
   Internet de las cosas, Virtualización, Infraestructura como servicio, Computación en el borde, Nube, LWM2M, Red local, Aplicación, Enrutador, Maquina virtual
   


