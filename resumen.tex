% +--------------------------------------------------------------------+
% | Copyright Page
% +--------------------------------------------------------------------+

\newpage

\thispagestyle{empty}

\begin{center}

{\bf \Huge Resumen en castellano}

  \end{center}
\vspace{1cm}

El presente trabajo describe la implementación de {\sc Epfiot} ({\em Edge Provisioning For Internet of Things}), una \textit{appliance} desarrollada para contemplar casos de Infraestructura como Servicio (IaaS, {\em Infrastructure as a service}) en el marco de \textit{edge computing}, y específicamente diseñada y preparada para interactuar con dispositivos IoT usando aceleradores {\em hardware} de propósito específico en el ámbito del aprendizaje automático.

En un mundo gobernado por el paraditma \textit{cloud computing}, surgen dudas sobre si es estrictamente necesario llevar a cabo un envío de todos los datos que recogen los dispositivos a Internet, afrontando la latencia que esto supone. 
%
{\sc Epfiot} persigue, a cambio de un uso contenido de recursos, ofrecer una  infraestructura completa para llevar a cabo un procesamiento integral de datos obtenidos desde dispositivos en la propia red local, facilitando máquinas en el borde ({\em edge computing} potencialmente equipadas con aceleradores de propósito específico para aprendizaje automático que realizan inferencia sobre los datos de forma rápida, reduciendo la latencia asociada a la transferencia a grandes centros de datos centralizados o distribuidos, pero en cualquier caso remotos.

La aplicación desarrollada ofrece una interfaz \texttt{GraphQL} que posibilita el desarrollo de un ecosistema IoT de infraestructura en el borde usando la virtualización de Linux ({\tt kvm}) y otras tecnologías emergentes como LwM2M facilitando la configuración de los dispositivos.

\vspace{1cm}

% +--------------------------------------------------------------------+
% | On the line below, repla	ce Fecha
% |
% +--------------------------------------------------------------------+

\begin{center}

{\bf \Large Palabras clave}

   \end{center}

   \vspace{0.5cm}
   
   Internet de las cosas, Virtualización, Infraestructura como servicio, Computación en el borde, Nube, LWM2M, Red local, Aplicación, Enrutador, Maquina virtual
   


